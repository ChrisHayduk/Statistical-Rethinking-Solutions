\documentclass[12pt]{article}\usepackage[]{graphicx}\usepackage[]{color}
%% maxwidth is the original width if it is less than linewidth
%% otherwise use linewidth (to make sure the graphics do not exceed the margin)
\makeatletter
\def\maxwidth{ %
  \ifdim\Gin@nat@width>\linewidth
    \linewidth
  \else
    \Gin@nat@width
  \fi
}
\makeatother

\definecolor{fgcolor}{rgb}{0.345, 0.345, 0.345}
\newcommand{\hlnum}[1]{\textcolor[rgb]{0.686,0.059,0.569}{#1}}%
\newcommand{\hlstr}[1]{\textcolor[rgb]{0.192,0.494,0.8}{#1}}%
\newcommand{\hlcom}[1]{\textcolor[rgb]{0.678,0.584,0.686}{\textit{#1}}}%
\newcommand{\hlopt}[1]{\textcolor[rgb]{0,0,0}{#1}}%
\newcommand{\hlstd}[1]{\textcolor[rgb]{0.345,0.345,0.345}{#1}}%
\newcommand{\hlkwa}[1]{\textcolor[rgb]{0.161,0.373,0.58}{\textbf{#1}}}%
\newcommand{\hlkwb}[1]{\textcolor[rgb]{0.69,0.353,0.396}{#1}}%
\newcommand{\hlkwc}[1]{\textcolor[rgb]{0.333,0.667,0.333}{#1}}%
\newcommand{\hlkwd}[1]{\textcolor[rgb]{0.737,0.353,0.396}{\textbf{#1}}}%
\let\hlipl\hlkwb

\usepackage{framed}
\makeatletter
\newenvironment{kframe}{%
 \def\at@end@of@kframe{}%
 \ifinner\ifhmode%
  \def\at@end@of@kframe{\end{minipage}}%
  \begin{minipage}{\columnwidth}%
 \fi\fi%
 \def\FrameCommand##1{\hskip\@totalleftmargin \hskip-\fboxsep
 \colorbox{shadecolor}{##1}\hskip-\fboxsep
     % There is no \\@totalrightmargin, so:
     \hskip-\linewidth \hskip-\@totalleftmargin \hskip\columnwidth}%
 \MakeFramed {\advance\hsize-\width
   \@totalleftmargin\z@ \linewidth\hsize
   \@setminipage}}%
 {\par\unskip\endMakeFramed%
 \at@end@of@kframe}
\makeatother

\definecolor{shadecolor}{rgb}{.97, .97, .97}
\definecolor{messagecolor}{rgb}{0, 0, 0}
\definecolor{warningcolor}{rgb}{1, 0, 1}
\definecolor{errorcolor}{rgb}{1, 0, 0}
\newenvironment{knitrout}{}{} % an empty environment to be redefined in TeX

\usepackage{alltt}
 
\usepackage[margin=1in]{geometry}
\usepackage{amsmath,amsthm,amssymb, mathtools}
\usepackage[T1]{fontenc}
\usepackage{lmodern}
\usepackage{fixltx2e}
\usepackage[shortlabels]{enumitem}
 
\newcommand{\N}{\mathbb{N}}
\newcommand{\R}{\mathbb{R}}
\newcommand{\Z}{\mathbb{Z}}
\newcommand{\Q}{\mathbb{Q}}
 
\newenvironment{theorem}[2][Theorem]{\begin{trivlist}
\item[\hskip \labelsep {\bfseries #1}\hskip \labelsep {\bfseries #2.}]}{\end{trivlist}}
\newenvironment{lemma}[2][Lemma]{\begin{trivlist}
\item[\hskip \labelsep {\bfseries #1}\hskip \labelsep {\bfseries #2.}]}{\end{trivlist}}
\newenvironment{exercise}[2][Exercise]{\begin{trivlist}
\item[\hskip \labelsep {\bfseries #1}\hskip \labelsep {\bfseries #2.}]}{\end{trivlist}}
\newenvironment{problem}[2][Problem]{\begin{trivlist}
\item[\hskip \labelsep {\bfseries #1}\hskip \labelsep {\bfseries #2.}]}{\end{trivlist}}
\newenvironment{question}[2][Question]{\begin{trivlist}
\item[\hskip \labelsep {\bfseries #1}\hskip \labelsep {\bfseries #2.}]}{\end{trivlist}}
\newenvironment{corollary}[2][Corollary]{\begin{trivlist}
\item[\hskip \labelsep {\bfseries #1}\hskip \labelsep {\bfseries #2.}]}{\end{trivlist}}
\newcommand{\textfrac}[2]{\dfrac{\text{#1}}{\text{#2}}}
\IfFileExists{upquote.sty}{\usepackage{upquote}}{}
\begin{document}

\title{Statistical Rethinking: Chapter 5 - Multivariate Linear Models}

\author{Chris Hayduk}
\date{\today}

\maketitle




\section{Easy}

\begin{problem}{5E1}
\text{ }\\
Which of the linear models below are multiple linear regressions?
\begin{enumerate}
	\item $\mu$\textsubscript{i} = $\alpha$ + $\beta$x\textsubscript{i}
	\item $\mu$\textsubscript{i} = $\beta$\textsubscript{x}x\textsubscript{i} + $\beta$\textsubscript{z}z\textsubscript{i}
	\item $\mu$\textsubscript{i} = $\alpha$ + $\beta$(x\textsubscript{i} - z\textsubscript{i})
	\item $\mu$\textsubscript{i} = $\alpha$ + $\beta$\textsubscript{x}x\textsubscript{i} + $\beta$\textsubscript{z}z\textsubscript{i}
\end{enumerate}
\end{problem}

Linear models 2 and 4 are multiple linear regressions.

\begin{problem}{5E2}
\text{ }\\
Write down a multiple regression to evaluate the claim: \textit{Animal diversity is linearly related to latitude, but only after controlling for plant diversity.} You just need to write down the model definition.
\end{problem}

\begin{center}
animal diversity\textsubscript{i} $\sim$ Normal($\mu$\textsubscript{i}, $\sigma$)\\
$\mu$\textsubscript{i} = $\beta$\textsubscript{latitude}latitude\textsubscript{i} + $\beta$\textsubscript{diversity}diversity\textsubscript{i}\\
$\beta$\textsubscript{latitude} $\sim$ Normal(0, 10)\\
$\beta$\textsubscript{diversity} $\sim$ Normal(0, 10)\\
$\sigma$ $\sim$ Uniform(0, 10)
\end{center}

\begin{problem}{5E3}
\text{ }\\
Write down a multiple regression to evaluate the claim: \textit{Neither amount of funding nor size of laboratory is by itself a good predictor of time to PhD degree; but together these variables are both positively associated with time to degree.} Write down the model definition and indicate which side of zero each slope parameter should be on.
\end{problem}

\begin{center}
time\textsubscript{i} $\sim$ Normal($\mu$\textsubscript{i}, $\sigma$)\\
$\mu$\textsubscript{i} = $\beta$\textsubscript{lab size}lab size\textsubscript{i} + $\beta$\textsubscript{funding}funding\textsubscript{i}\\
$\beta$\textsubscript{lab size} $\sim$ Normal(0, 10)\\
$\beta$\textsubscript{funding} $\sim$ Normal(0, 10)\\
$\sigma$ $\sim$ Uniform(0, 10)
\end{center}

Both parameters should have slopes greater than zero since the problem specifies that "together the variables are both positively associated with time to degree".

\begin{problem}{5E4}
\text{ }\\
Suppose you have a single categorical predictor with 4 levels (unique values), labeled A, B, C, and D. Let A\textsubscript{i} be an indicator variable that is 1 where case \textit{i} is in category A. Also suppose B\textsubscript{i}, C\textsubscript{i}, and D\textsubscript{i} for the other categories. Now which of the following linear models are inferentially equivalent ways to include the categorical variable in a regression? Models are inferentially equivalent when it's possible to compute one posterior distribution from the posterior distribution of another model.
\begin{enumerate}
	\item $\mu$\textsubscript{i} = $\alpha$ + $\beta$\textsubscript{A}A\textsubscript{i} + $\beta$\textsubscript{B}B\textsubscript{i} + $\beta$\textsubscript{D}D\textsubscript{i}
	\item $\mu$\textsubscript{i} = $\alpha$ + $\beta$\textsubscript{A}A\textsubscript{i} + $\beta$\textsubscript{B}B\textsubscript{i} + $\beta$\textsubscript{C}C\textsubscript{i} + $\beta$\textsubscript{D}D\textsubscript{i}
	\item $\mu$\textsubscript{i} = $\alpha$ + $\beta$\textsubscript{B}B\textsubscript{i} + $\beta$\textsubscript{C}C\textsubscript{i} + $\beta$\textsubscript{D}D\textsubscript{i}
	\item $\mu$\textsubscript{i} = $\alpha$\textsubscript{A}A\textsubscript{i} + $\alpha$\textsubscript{B}B\textsubscript{i} + $\alpha$\textsubscript{C}C\textsubscript{i} + $\alpha$\textsubscript{D}D\textsubscript{i}
	\item $\mu$\textsubscript{i} = $\alpha$\textsubscript{i}(1 - B\textsubscript{i} - C\textsubscript{i} - D\textsubscript{i}) + $\alpha$\textsubscript{B}B\textsubscript{i} + $\alpha$\textsubscript{C}C\textsubscript{i} + $\alpha$\textsubscript{D}D\textsubscript{i}
\end{enumerate}
\end{problem}

Models 1, 3, 4, and 5 are all inferentially equivalent.

\end{document}
